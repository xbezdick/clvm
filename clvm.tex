\documentclass{beamer}
\mode<presentation>
{
  \usetheme{Warsaw}
  \setbeamercovered{transparent}
}
\usepackage[slovak]{babel}
\usepackage[utf8]{inputenc}
\usepackage[T1]{fontenc}
\usepackage{tgpagella}
\usepackage{cmap}
\usepackage{graphicx}

\title[CLVM]
{Clustered Logical Volume Manager}

\author[Bezdička L.]
{Lukáš Bezdička}
\institute[FI MUNI]
{ Fakulta informatiky\\
  Masarykova univerzita}
\date[2011] 
{PV208}
\subject{FI MUNI PV208}
\pgfdeclareimage[height=0.5cm]{university-logo}{fi-logo}
\logo{\pgfuseimage{university-logo}}

\begin{document}

\begin{frame}
  \titlepage
\end{frame}

\begin{frame}
  \frametitle{Obsah}
  \tableofcontents
\end{frame}

\section{LVM}

\subsection{Prečo LVM}

\begin{frame}
  \frametitle{Výhody LVM}

  \begin{itemize}
  \item Abstrakcia nad HW.
  \item Volume management na vytváranie logických diskov.
  \item Logické disky nie sú späté s vlastnosťami fyzických diskov.
  \item Presúvanie dát medzi fyzickými diskami za behu, kopírovanie dát na viaceré disky (mirroring), snapshoty a ďalšie.
  \end{itemize}
\end{frame}

\subsection{Architektúra LVM}

\begin{frame}
  \frametitle{Architektúra LVM}
  Physical volume > Volume group > Logical volume
\end{frame}

\begin{frame}
PVG (Physical Volume Group) > PV (Physical Volume) > PE - physical extends (časti diskov)
\end{frame}

\begin{frame}
VG (Volume Group) > LV (Logical Volume) > LE - logical extends
\end{frame}

\begin{frame}
Logické časti LE sa mapujú na PE v pomere 1 : 1..N.
\end{frame}

\subsection{Ako?}

\begin{frame}
  \frametitle{Label}
  Label pozostáva z:
  \begin{itemize}
  \item názov
  \item uuid
  \item veľkosť v bitoch
  \item lokácia metadát
  \end{itemize}
   Zvyk druhý 512-byte sektor, ale môže byť na ktoromkoľvek z prvých 4-och sektorov.
\end{frame}

\begin{frame}
  \frametitle{Metadata}
  Metadáta - ASCII dáta o konfigurácii, ktoré sa udržujú rovnaké na každej PV pre každú VG v priestore pre metadáta.
  Štandardne existuje viacero kópií metadát, pri veľa fyzickych diskoch vo VG sú zálohy neefektívne => \texttt{pvcreate --metadatacopies 0}.
  
  Ako sa dostať k metadátam:
  
  \texttt{strings /dev/disk | grep "\# Generated by LVM" -A 2048 > output.txt}
\end{frame}

\subsection{Is there a Problem Officer?}

\begin{frame}
  \frametitle{Cluster}
  \includegraphics[width=2cm]{problem.png}
  
  
Cluster, kde dochádza k používaniu disku naraz viacerými strojmi.

Ak pristupuje vždy len jeden stroj a ide o HA Cluster > HaLVM.
\end{frame}

\section{CLVM}

\subsection{Clustered Logical Volume Manager}

\begin{frame}
\frametitle{Clustered Logical Volume Manager}
Rozšírenie LVM --- pridáva uzamykanie PE počas prístupu na logický disk atď.
\note{atď - clustered locking system}
\end{frame}
\subsection{Ako na to?}
\begin{frame}
  \frametitle{Ako na to?}
  \begin{itemize}
  \item Bežiaci cluster.
  \item Spustiť \textbf{clvmd} - \texttt{chkconfig clvmd on;...}
  \item Zaistit všetkým nódam prístup k PV.
  \item Inak je práca s LVM skoro rovnaká ako bez clustera.
  \end{itemize}
\end{frame}

\begin{frame}
\frametitle{Nutná zmena uzamykania lvm.}
\texttt{/sbin/lvmconf --enable-cluster}
alebo úpravou \texttt{/etc/lvm/lvm.conf} zmeniť zamykanie.
\texttt{locking = 3} pre zabudovanú podporu clustrového zamykania.
\end{frame}

\begin{frame}
 \texttt{\#cman\_tool services} na kontrolu, či nám beží \textbf{clvmd}.
 
 
 
 \texttt{DLM (Distributed Lock Manager) lockspaces - clvmd}
\end{frame}

\begin{frame}
\frametitle{Mirroring}
\textbf{cmirrord} pokiaľ chceme mirroring!

Potom uz len PVs > VGs > \texttt{lvcreate ... -m1 ...}

\texttt{-m1} mirroruje na jeden disk \texttt{-m2} na dva.
\end{frame}

\end{document}


%Particie > typy. Zistit efigpt > udev (fdisk =[ parted]).
